\documentclass{../template/Phyport}

\usepackage{hyperref}
\hypersetup{
    colorlinks=true,
}

\exname{弗兰克赫兹实验} %实验名称
\extable{} %实验桌号
\instructor{} %指导教师
\class{} %班级
\name{侯国玉} %姓名
\stuid{32114514} %学号

\nyear{2025} %年
\nmonth{} %月
\nday{} %日
\nweekday{} %星期几

\redate{似掉了。} %如有实验补做,补做日期
\resitu{} %情况说明:

\begin{document}
\setcounter{page}{0}
\makecover

\section{预习报告(10分)}
(注:将已经写好的“物理实验预习报告”内容拷贝过来)

\subsection{实验综述(5分)}
(自述实验现象、实验原理和实验方法,包括必要的光路图、电路图、公式等。不超过500字。)

\subsection{实验重点(3分)}
(简述本实验的学习重点,不超过100字。)

\subsection{实验难点(2分)}
(简述本实验的实现难点,不超过100字。)

\section{原始数据(20分)}
(将有老师签名的“自备数据记录草稿纸”的扫描或手机拍摄图粘贴在下方,完整保留姓名,学号,教师签字和日期。)

\section{结果与分析(60分)}
\subsection{数据处理与结果(30分)}
(列出数据表格、选择适合的数据处理方法、写出测量或计算结果。)

\subsection{误差分析(20分)}
(运用测量误差、相对误差或不确定度等分析实验结果,写出完整的结果表达式,并分析误差原因。)

\subsection{实验探讨(10分)}
(对实验内容、现象和过程的小结,不超过100字。)

\section{思考题(10分)}
(解答教材或讲义或老师布置的思考题,请先写题干,再作答。)

{\fangsong \noindent \textbf{注意事项:}

\begin{enumerate}[label=\arabic*.]
    \item 用PDF格式上传“预习报告”,文件名:学生姓名+学号+实验名称+周次。
    \item “预习报告”必须递交在“学在浙大”本课程内对应实验项目的“作业”模块内。
    \item “预习报告”还须拷贝到“实验报告”中(以便教师批改)。
    \item “大学物理实验”课程选做实验使用本“预习报告”;必做实验无须写预习报告,前往“学在浙大”完成预习测试即可。
\end{enumerate}}
\begin{center}
    \fbfangsong{浙江大学物理实验教学中心制}
\end{center}
\end{document}
