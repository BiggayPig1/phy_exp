\documentclass[]{../template/Report}%方括号内写yuxi即生成预习报告\documentclass[yuxi]{../template/Report}
\settemplatedir{../template/}%设置模板路径

\exname{} %实验名称
\extable{} %实验桌号
\instructor{} %指导教师
\class{} %班级
\name{} %姓名
\stuid{} %学号

\nyear{} %年
\nmonth{} %月
\nday{} %日
\nweekday{} %星期几,e.g. \nweekday{三}
\daypart{}%上午/下午

\redate{} %如有实验补做,补做日期
\resitu{} %情况说明:

\begin{document}
\maketitle%输出封面

\section{预习报告(10分)}
(注:将已经写好的“物理实验预习报告”内容拷贝过来)

\subsection{实验综述(5分)}
(自述实验现象、实验原理和实验方法,包括必要的光路图、电路图、公式等。不超过500字。)

\subsection{实验重点(3分)}
(简述本实验的学习重点,不超过100字。)

\subsection{实验难点(2分)}
(简述本实验的实现难点,不超过100字。)

\begin{fullreportonly}
\section{原始数据(20分)}
(将有老师签名的“自备数据记录草稿纸”的扫描或手机拍摄图粘贴在下方,完整保留姓名,学号,教师签字和日期。)

\section{结果与分析(60分)}
\subsection{数据处理与结果(30分)}
(列出数据表格、选择适合的数据处理方法、写出测量或计算结果。)

\subsection{误差分析(20分)}
(运用测量误差、相对误差或不确定度等分析实验结果,写出完整的结果表达式,并分析误差原因。)

\subsection{实验探讨(10分)}
(对实验内容、现象和过程的小结,不超过100字。)

\section{思考题(10分)}
(解答教材或讲义或老师布置的思考题,请先写题干,再作答。)
\end{fullreportonly}
\insertnotes
\end{document}